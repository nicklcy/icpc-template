\subsubsection{伯努利数}
\noindent 伯努利数满足
\begin{equation*}
    B_0 = 1, \sum_{j=0}^m \binom{m+1}{j} B_j = 0 ~ (m > 0).
\end{equation*}
\noindent 等式两边同时加上 $B_{m+1}$,并设 $n = m - 1$,得
\begin{equation*}
\sum_{i=0}^n \binom{n}{i} = [n = 1] + B_n
\end{equation*}
\noindent 设 $\hat{B}(x) = \sum_{i=0}^\infty B_i \cdot \frac{x^i}{i!}$,则
\begin{equation*}
\hat{B}(x) e^x = x + \hat{B}(x) \Rightarrow \hat{B}(x) = \frac{x}{e^x - 1} \\
\end{equation*}
\begin{equation*}
\begin{aligned}
& 0^k + 1^k + \ldots + n^k \\
=& ~ k! [x^k] \frac{e^{(n+1)x}-1}{x} \cdot \hat{B}(x) \\
=& ~ k! \sum_{i=0}^k \frac{B_i}{i!} \cdot \frac{(n+1)^{k-i+1}}{(k-i+1)!} \\
=& ~ \frac{1}{k+1} \sum_{i=0}^k \binom{k+1}{i} B_i \cdot (n+1)^{k-i+1}
\end{aligned}
\end{equation*}

\subsubsection{第一类斯特林数}
\noindent 记 $S_1(n, k)$ 为将 $n$ 个不同元素分为 $k$ 个环排列的方案数. 由组合意义得,
$$
S_1(n, k) = S_1(n - 1, k - 1) + (n - 1) S_1(n - 1, k)
$$
$$
x^{\overline{n}} = \sum_{i=0}^n S_1(n, i) x^i
$$
$$
x^{\underline{n}} = \sum_{i=0}^n (-1)^{n-i} S_1(n, i) x^i
$$
$$
\sum_{i=0}^n S_1(n, i) x^i = \prod_{i=0}^{n-1} (x+i)
$$

\noindent 注意最后等式的右半部分,可以使用递增 + 点值平移 $O(n \log n)$ 求出第
$n$ 行斯特林数.

\subsubsection{第二类斯特林数}
\noindent 记 $S_2(n, k)$ 为将 $n$ 个不同元素分至 $k$ 个相同的盒子(每个盒子至
少一个元素)的方案数. 由组合意义得,
$$
S_2(n, k) = S_2(n - 1, k - 1) + k S_2(n - 1, k)
$$
$$
x^n = \sum_{i=0}^n S_2(n, i) x^{\underline{i}}
$$
$$
S_2(n, k) = \sum_{i=0}^k (-1)^{k-i} \binom{k}{i} i^n
$$
$$
\frac{S_2(n, k)}{k!} = \sum_{i=0}^k \frac{i^n}{i!} \cdot \frac{(-1)^{k-i}}
{(k-i)!}
$$
\noindent 是一个卷积的形式,可以 FFT 求出某一行第二类斯特林数.

\subsubsection{斯特林反演}
\begin{equation*}
    \begin{aligned}
        x^n &= \sum_{i=0}^n S_2(n, i) x^{\underline{i}} \\
        &= \sum_{i=0}^n S_2(n, i) \sum_{j=0}^i (-1)^{i-j} S_1(i, j) x^j \\
        &= \sum_{i=0}^n x^i \sum_{j=i}^n (-1)^{j-i} S_2(n, j) S_1(j, i)
    \end{aligned}
\end{equation*}
\noindent 设
\begin{equation*}
    g_n = \sum_{i=0}^n S_2(n, i) f_i,
\end{equation*}
\noindent 则
\begin{equation*}
    f_n = \sum_{i=0}^n (-1)^{n-i} S_1(n, i) g_i.
\end{equation*}

\subsubsection{Burnside 引理}
设置换群为 $G$,染色集合为 $X$.

若染色 $x \in X$ 在置换 $f$ 的作用下得到染色 $y \in X$,则称 $x, y$ 等价.
由置换群的定义,我们可以得到等价类,使得等价类内任意两个染色等价.

设 $X^g (g \in G)$ 表示在置换 $g$ 下的不动点,即
$$
X^g = \{x \mid x \in X, gx = x\}.
$$
则等价类个数
$$
|X / G| = \frac{1}{|G|} \sum_{g \in G} |X^g|.
$$

例 LOJ 6538 烷基计数,对于一棵有根树,每个节点至多三个儿子,且这些儿子排列同
构. 求有多少个 $n$ 个节点的等价类.

考虑其生成函数 $f(x)$.
根节点有 $3$ 个儿子(儿子可以为空,因为循环同构,我们不需讨论 $0, 1, 2$ 个儿子
的情况),排列的置换群有 $6$ 种,其中 $(1,2,3)$ 染色方案数为 $f(x)^3$,
$(1,3,2),(2,1,3),(3,2,1)$ 染色方案为 $f(x^2)f(x)$,$(2,3,1),(3,1,2)$ 染色方案
为 $f(x^3)$. 所以
$$
f(x) = x \times \frac{f(x)^3+3f(x^2)f(x)+2f(x^3)}{6}+1.
$$
牛顿迭代.
