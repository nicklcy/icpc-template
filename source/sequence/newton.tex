\noindent \textbf{问题描述:}给出多项式$G(x)$,求解多项式$F(x)$满足:
\[G(F(x)) \equiv 0 \pmod {x^n}\]
答案只需要精确到$F(x) \bmod {x^n}$即可。\par
\noindent \textbf{实现原理:}考虑倍增,假设有:
\[G(F_t(x)) \equiv 0 \pmod{x^{2^t}}\]
对$G(F_{t + 1}(x))$在模$x^{2^{t+1}}$意义下进行Taylor展开:
\[G(F_{t + 1}(x)) \equiv G(F_t(x)) + \dfrac{G'(F_t(x))}{1!}(F_{t + 1}(x) - F_t(x)) \pmod{x^{2^{t+1}}}\]
那么就有:
\[F_{t + 1}(x) \equiv F_t(x) - \dfrac{G(F_t(x))}{G'(F_t(x))} \pmod{x^{2^{t+1}}}\]
\noindent \textbf{注意事项:}$G(F(x))$的常数项系数必然为0,这个可以作为求解的
初始条件。
\par
\noindent \textbf{多项式求逆}
\noindent \textbf{原理:}令$G(x) = x \times A - 1$(其中$A$是一个多项式系数),根据牛顿迭代法有:
\begin{displaymath}
\begin{split}
F_{t + 1}(x) &\equiv F_t(x) - \dfrac{F_t(x) \times A(x) - 1}{A(x)} \\
    &\equiv 2F_t(x) - F_t(x)^2 \times A(x)\pmod{x^{2^{t+1}}}
\end{split}
\end{displaymath}
\noindent \textbf{注意事项:}
\begin{enumerate}
	\item $F(x)$的常数项系数必然不为0,否则没有逆元;
	\item 复杂度是$O(n \log n)$但是常数比较大($10^5$大概需要0.3秒左右);
	\item 传入的两个数组必须不同,但传入的次数界没有必要是2的次幂;
\end{enumerate}
\textbf{多项式取指数和对数}
\noindent \textbf{作用:}给出一个多项式$A(x)$,求一个多项式$F(x)$满足$e^A(x) - F(x) \equiv 0 \pmod{x^n}$。\par
\noindent \textbf{原理:}令$G(x) = \ln x - A$(其中$A$是一个多项式系数),根据牛顿迭代法有:
\[F_{t + 1}(x) \equiv F_t(x) - F_t(x)(\ln {F_t(x)} - A(x)) \pmod{x^{2^{t+1}}}\]
求$\ln {F_t(x)}$可以用先求导再积分的办法,即:
\[\ln A(x) = \int \dfrac{F'(x)}{F(x)}~\mathrm{d}x\]
多项式的求导和积分可以在$O(n)$的时间内完成,因此总复杂度为$O(n \log n)$。\par
\noindent \textbf{应用:}加速多项式快速幂。\par
\noindent \textbf{注意事项:}
\begin{enumerate}
	\item 进行$\log$的多项式必须保证常数项系数为1,否则必须要先求出$\log a[0]$是多少;
	\item 传入的两个数组必须不同,但传入的次数界没有必要是2的次幂;
	\item 常数比较大,$10^5$的数据求指数和对数分别需要0.37s和0.85s左右的时间,注意这里{memset}几乎不占用时。
\end{enumerate}
